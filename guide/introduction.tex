Clojure, in a nutshell, is a functional programming language that runs on-top of the Java Virtual Machine (JVM).
%TODO: Mention the discussion of the term ``functional programming'' here (incl. reference).
In this guide I will try to introduce Clojure in a pragmatic, hands-on fashion.
There will be some discussion of the background etc. but the focus is on working with Clojure and its tools.

I assume that you are reading this because you are interested in Clojure or functional programming in general.
However, I like to make just some short remarks just in case you are not sure if you should read on.
%So, I will not try to convince you to use Clojure or explain how great it, in my opinion, is.
Personally, I like Clojure very much;
it is a powerful language, offers great tools, and allows usage of a vast range of pre-existing libraries that help you to develop quickly and efficiently.
Additionally, if you are, for now, only used to procedural (C or Turbo Pascal $<$ 7.0) or object-oriented programming (C++, Java, etc.) the functional programming paradigm opens up an entirely new perspective on how computational or software engineering tasks can be solved.
Using a Lisp (Clojure is a Lisp dialect.) you will find much less conventions built into the language that constrain you during development.
%TODO: Add vref?
While I really think Clojure is great I won't stress you too much and try to proceed right away.
I hope you will discover the usefulness of Clojure and its power as you go on with this guide.

Throughout this guide, the Clojure specific terminology will be introduced step by step.
% If you are interested already now in a little bit more background information about Clojure please see \vref{TODO}.
In this guide, you will, amongst others, learn about things like:
what it means to use a Lisp,
what ``homoiconic'' means,
functions being first-class,
what the Read Eval Print Loop (REPL) is,
and how to make use of all this.
Besides these fundamentals, topics like
editor support for writing Clojure code,
test-driven development using Clojure,
or automation for efficiently working with Clojure projects in day-to-day development will be covered as well.

