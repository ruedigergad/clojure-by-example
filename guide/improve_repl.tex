\chapter{Sweeten the REPL}
\label{sec:improve_repl}
At this stage you should already have a working Clojure installation.
To be honest, the steps we do now are optional.
Still, I highly recommend doing them now.

If you are impatient and want to hack right away you can skip this section and jump directly to section \vref{sec:first_repl_steps}.
You can come back and do the steps described here at any time later on.

Right now, you can already launch a Clojure REPL.
However, it is not very convenient to work with it as it is at the moment.
We will change that here.
We will modify the start script to allow much more convenient interaction with the REPL.

This modification will add a ``history''.
With this history you can scroll back and forth in ``old'' things that you had already entered earlier.
This is similar to what you might already be used to in a shell or terminal.

There are different options for realizing this.
Depending on which option you choose, you will also be able to also search in your history using the key combination ``\texttt{CTRL+R}'', like in the Bourne-again Shell (bash).
In the following I will describe two options:
\begin{itemize}
  \item using \texttt{rlwrap}
  \item and using \texttt{jline}.
\end{itemize}

\section{\texttt{rlwrap} Method}
\label{sec:rlwrap}
\texttt{rlwrap} is a wrapper for \texttt{readline}.
This is the recommended method when you are using a Linux distribution as operating system.
Reasons are that this is more powerful and it is easier to use as compared to using \texttt{jline}.

To use \texttt{rlwrap} two steps are necessary:
installing \texttt{rlwrap} and modifying the Clojure start script.

\subsection{Install \texttt{rlwrap}}
First of all, you need to install \texttt{rlwrap} with your favorite package manager.
Below Listings \vrefrange{lst:rlwrap_install_debian}{lst:rlwrap_install_gentoo} exemplarily show the installation of \texttt{rlwrap} for some Linux distributions.

\begin{lstlisting}[label=lst:rlwrap_install_debian, caption=Installation of \texttt{rlwrap} on Debian and Debian-based distributions like Ubuntu or Kubuntu]
apt-get install rlwrap
\end{lstlisting}

\begin{lstlisting}[label=lst:rlwrap_install_fedora, caption=Installation of \texttt{rlwrap} on Fedora or Red Hat]
yum install rlwrap
\end{lstlisting}

\begin{lstlisting}[label=lst:rlwrap_install_gentoo, caption=Installation of \texttt{rlwrap} on Gentoo or Funtoo]
emerge rlwrap
\end{lstlisting}

\subsection{Modify Clojure Start Script}
Finally, the Clojure start script needs to be modified.
This modification is very simple.
It is sufficient to add ``\texttt{rlwrap}'' in front of the ``\texttt{java}'' call in the startup script.

In the case of the custom start script as created in section \vref{sec:manual_install} the new start script looks as shown in Listing \vref{lst:improved_start_script}.

\begin{lstlisting}[label=lst:improved_start_script, caption=Modified Start Script with \texttt{rlwrap}]
#!/bin/sh
rlwrap java -jar <PATH_TO_FILE>/clojure-<VERSION>.jar
\end{lstlisting}

If Clojure was installed via the package manager as described in section \vref{sec:install_via_package_manager} there are two options:
Firstly, a custom start script as described in section \vref{sec:manual_install} can be created or the existing start script can be modified.

When modifying the existing start script at first the start script must be located.
You can locate the start script, e.\,.g, with \texttt{which}.
In Listing \vref{lst:which_clojure} I show what it looks like on my computer.

\begin{lstlisting}[label=lst:which_clojure, caption=Locating the Distribution Supplied Clojure Start Script]
[rc@colin ~]$ which clojure
/bin/clojure
\end{lstlisting}

Now that we know the Clojure start script is ``\texttt{/bin/clojure}'' we can edit this file and add the \texttt{rlwrap} call.
The modification is analogous to the modification of the custom start script as described above;
simply ``\texttt{rlwrap}'' must be added in front of ``\texttt{java}''.
I am using Fedora, the modified start script is shown in Listing \vref{lst:fedora_improved_clojure_script}.

\begin{lstlisting}[label=lst:fedora_improved_clojure_script, caption=Modified Clojure Start Script on Fedora 17]
#!/bin/bash
exec rlwrap java ${JAVA_OPTS} -jar /usr/share/java/clojure.jar "$@"
\end{lstlisting}

\section{\texttt{jline} Method}
Similarly to \texttt{rlwrap}, \texttt{jline} is a wrapper that adds some convenience to textual input user interfaces.
One difference is that \texttt{jline} is written in Java.
This has the big advantage that it is operating system independent and also works with MacOS or Windows.
On the other hand, \texttt{jline} is not as powerful as \texttt{rlwrap}.
However, using \texttt{jline} is still a huge improvement as compared to using the REPL without it.

In the following, I will deliberately skip instructions for installing and using \texttt{jline} via a package manager.
Reason for this is that I assume that if you are using a package manager you are going with the \texttt{rlwrap} method as described in section \vref{sec:rlwrap}.

Similarly to the \texttt{rlwrap} method, using jline requires the two steps of getting it and integrating it into the script.

\subsection{Get \texttt{jline}}
You can download \texttt{jline} from \url{http://sourceforge.net/projects/jline/files/}.
Simply download the \texttt{*.zip} file, extract it, and store the \texttt{*.jar} file somewhere on your computer.
Similarly to the manual Clojure installation method, a good place for putting the \texttt{*.jar} (on Linux) could be \texttt{/usr/loca/lib}.

\subsection{Modify the Start Script}
After \texttt{jline} had been downloaded and saved the start script needs to be modified.
I will only cover modifying the custom start script here;
As already stated, I assume that you are not going with this method when you used the package manager for installing Clojure and while, hence, favor the \texttt{rlwrap} approach.
In Listing \vref{lst:jline_improved_clojure_script} the modified version of the script as created in section \vref{sec:manual_install} is shown.

\begin{lstlisting}[label=lst:jline_improved_clojure_script, caption=Clojure Start Script Modified to Use \texttt{jline}]
#!/bin/sh
CLASSPATH=$CLASSPATH:<PATH_TO_JLINE>/jline.jar:<PATH_TO_CLOJURE>/clojure-<VERSION>.jar java jline.ConsoleRunner clojure.main
\end{lstlisting}


