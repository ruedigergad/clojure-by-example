\chapter{First Steps in the REPL}
\label{sec:first_repl_steps}
In this chapter we will execute the first, simple Clojure code.
We do this by interactively playing around and experimenting with the \bfix{Read Eval Print Loop} (\bfix{REPL}).

Up to now we only did some preparations.
Now it is the time to get your hands on real Clojure code.

We will start very simple and will slowly increase the complexity.
During this chapter we will introduce the very basics of Clojure like data types, how to ``call'' functions,
%(In Clojure functions are said to be ``evaluated''.)
basic data structures, how to define ``variables'', how to create our own functions, and will already mention some Clojure specific terminology.

Do not be disappointed, afraid, or confused if you don't grasp or memorize everything right away.
I will try to repeat the important things later on in following chapters.
Here, the primary intent is that you get started with actually playing around with Clojure.
I just introduce some terminology already here so that you get used to this terminology.
By the way, repetition is considered a good way for learning. ;)

\section{Hello \cancel{World} REPL}
As the very first step we will continue with a slightly modified tradition when learning programming languages.
We will make the REPL say ``Hello'' to us.

So start the REPL by executing the respective start script as set up in chapter \vref{sec:get_clojure}.
In the REPL type ``\texttt{(println ''Hello REPL!``)}''.
The result should look like shown in Listing \vref{lst:hello_repl}.

\begin{lstlisting}[label=lst:hello_repl, caption=Say ``Hello'' REPL! -- ``Hello REPL!'']
[rc@colin clojure-by-example]$ clojure 
Clojure 1.4.0
user=> (println "Hello REPL!")
Hello REPL!
nil
user=> 
\end{lstlisting}

What happened here?
In the REPL you executed, more particularly evaluated, your first function (``\texttt{println}'').
Here, we can make different noteworthy observations:
firstly, how the REPL works and why it is called Read Eval Print Loop.
Secondly, what the REPL displays.
Additionally, you see that Clojure code is, most probably, very different to what you might be used to 
(If you haven't done much with Lisp or one of its dialects yet.\footnote{If you are new to Lisp-like languages be prepared, but not scared, to see lots of brackets. ;)}).

\section{How does the REPL work?}
Shown in Listing \vref{lst:hello_repl} is what the output of evaluating your very first function in the REPL looks like.
In the third line you can see the REPL prompt: ``\texttt{user=>}''.
At this prompt you can enter arbitrary Clojure code.

\begin{enumerate}
  \item In the first step, the REPL \textbf{reads} the code entered at the prompt (see Listing \vref{lst:hello_repl} line 3).
  \item Then it \textbf{evaluates} (In Clojure/Lisp terminology code is evaluated.) what had been read.
  \item Afterwards it \textbf{prints} the results (see Listing \vref{lst:hello_repl} line 4 and 5).
  \item Finally, it displays the prompt again and waits for input in order to perform the next \textbf{loop} iteration (see Listing \vref{lst:hello_repl}).
\end{enumerate}

So, this is, in a nutshell, the very simple and pragmatic reason why the Read Eval Print Loop or short ``REPL'' is called the way it is.

Please note that ``reading'' in this case is not what we humans usually associate with reading a text or in computing terms typically understand as ``reading'' a file from disk into memory.
In fact during the so called \bfix{read phase} much more may happen than just ``getting'' code from the REPL prompt or from a Clojure file.
The read phase is one of two phases during which code is processed in Clojure.
The second such phase is the so called \bfix{evaluation phase}.
The differentiation between these two phases will become important later when we start talking about what is known as ``macros''.
For now, we simply ignore the distinction between those phases for the sake of simplicity and rather proceed with some more basics.

\section{What does the REPL display?}
If you have read the previous section the answer is very simple:
the REPL prints out the result of evaluating the code put into it.

However, we have to be more accurate here.
Clojure functions may ``return'' ``something'' as result of their evaluation.
%TODO: What is the Clojure terminology for ``returning'' something?
This result is one of the things printed at the REPL.

On the other hand, we may have console output (usually stdout or stderr) that is printed as well.
In the above example we used the function ``\texttt{println}'' to write ``Hello REPL!'' to stdout.
This console output is shown in line 4 of Listing \vref{lst:hello_repl}.

The actual ``return value'' is then printed below this output in line 5.
Here the ``returned'' value is ``\bfix{nil}''.
\texttt{nil} is a special value that can be compared to ``\texttt{NULL}'' or ``\texttt{null}'' in other languages like C++ or Java respectively.

To get a better idea of this difference just enter some code that does not print anything but solely evaluates to a value.
In order to try this we will simply add two numbers by entering ``\texttt{(+ 19 23)}''.
The result is shown in Listing \vref{lst:simple_add}.

\begin{lstlisting}[label=lst:simple_add, caption=Simple Addition in the REPL]
user=> (+ 19 23)
42
\end{lstlisting}

\section{Clojure: A Lisp Dialect}
Clojure is a \bfix{Lisp} dialect.
This has certain implications on the way Clojure code is written and represented.
In this section we will look at some of the basic implications connected with using a Lisp-like language like Clojure.

What you might have noticed first is that everything is written encapsulated within brackets ``\texttt{(\ldots)}''.
To understand this, a little bit of background about Lisp is required.
Lisp is a shorthand for ``\bfix{list processing}'', and this is basically what all Lisp-like languages do: they process lists.

In Clojure, and many other Lisps, a \bfix{list} is denoted using round brackets: ``\texttt{(<1st element> <2nd element> <3rd element> \ldots)}''.
So, when you entered ``\texttt{(+ 19 23)}'' at the REPL what you did was to enter a list with first element ``\texttt{+}'', second element ``\texttt{19}'', and third element ``\texttt{23}''.

At this point, it is important to know that there is a special convention in Clojure (or Lisps in general) that defines that lists are treated specially.
This convention, essentially, says that the first element of a list is expected to be ``something'' that can be called or executed.
That ``something'' may be a \bfix{function}, a \bfix{macro}, or a \bfix{special form}.
For now, we will not look into the differences of these three;
at this point, it is sufficient to know that the first element of a list is expected to be ``callable'' or ``executable''.

To cross-check the above statement and to make yourself a little bit more comfortable with this convention just try to enter a list that has a first element that cannot be evaluated;
e.\,g., enter ``\texttt{(3 5 9)} at the REPL and try to evaluate this.
The result should look similar to what is shown in Listing \vref{lst:list_error}.

\begin{lstlisting}[label=lst:list_error, caption=Result of an Attempt to Evaluate a List with a Non-evaluable First Item]
user=> (3 5 9)
ClassCastException java.lang.Long cannot be cast to clojure.lang.IFn  user/eval19 (NO_SOURCE_FILE:11)
user=>
\end{lstlisting}

Here, you can see that the attempt to evaluate a list that does not contain an item that can be executed in the very first position fails with an exception.
More particularly, the exception states that and object of type ``\texttt{java.lang.Long}'' cannot be cast to ``\texttt{clojure.lang.IFn}''.
This is because Clojure tries to cast the first element in the list to ``\texttt{clojure.lang.IFn}'' in order to call or execute it.
However, the first element in this list is ``\texttt{3}'' which is of type ``\texttt{java.lang.Long}''.
Thus, the exception is thrown as shown in Listing \vref{lst:list_error}.

This special way lists are treated has some consequences when entering or coding lists.
We will come back to this later when it gets important.

Another thing you can observe here is the notion of Clojure being \bfix{homoiconic}.
%TODO: Explain where the term ``homoiconic'' originates from?
Homoiconic, essentially, means that the source code is written in the data structures as provided by the language itself.
So Clojure source code, basically, consists of Clojure data structures.
In the very simple example here you could already see that the basic way to formulate expressions is done via lists.
Later on we will also see how different data structures are used in different places.
Being homoiconic is a very neat feature that, as we will see later, allows very powerful operations.

When you are new to Clojure you might be surprised by some of its conventions.
However, keep in mind that there are very few conventions ``built into'' Clojure.
If you compare this to a different language like C++ there are much more conventions built into the language which makes it much less flexible than Lisp-dialects like Clojure.
Conventions defined in C++ are, e.\,g., keywords like ``\texttt{class}'', ``\texttt{void}'', ``\texttt{int}'', or ``\texttt{new}'' but also syntactical constructs like how class or method definitions are structured (using combinations of round and curly brackets with semi-colons as deliminators etc.).

Clojure, on the other hand, allows you to freely change it in nearly any way you want.
You will be able to, essentially, create your own programming language that is tailored to fit the needs of a specific issue or task.
This is what is a also known as \bfix{domain specific language} (\bfix{DSL}).
However, for now, we do not go into such advanced topics like DSLs.

Yet another thing you might have noticed is that the ``operation'' is expected to be the first element of a list.
Because Clojure expects the ``operation'' or ``operator'' to be the first element, it is also said to use ``\bfix{prefix notation}''.
This is called ``prefix notation'' because the ``operator'' is ``in front of'' (which roughly translates to ``pre'' or ``prefix'' in Latin) the operands; e.\,g., ``\texttt{(+ 1 2)}''.
Prefix notation might be a little different to what you are already used to from other languages like Python, C, C++, or Java, for at least mathematical operations.
For mathematical expressions, these languages use what is referred to as ``\bfix{infix notation}''.
The name of ``infix notation'' essentially describes that the operator is ``in between'' the operands; e.\,g., ``\texttt{1 + 2}''. 

Prefix notation allows to have more than two operators per operation.
E.\,g., with prefix notation you can do the following: ``\texttt{(+ 1 2 3 4 5)}''.
With infix notation you would need to chain the operator like this: ``\texttt{1 + 2 + 3 + 4 + 5}''.

\section{Help to Help Yourself}
%``Give someone a fish and you'll feed him for one day. Teach him to fish and you'll feed him a life long.'' -- Original Author Unknown.
With this document I try to provide a guide to Clojure.
However, you should be ready and prepared to search for answers on your own as well.
Hence, I introduce some ways for finding help for Clojure.

\subsection{Built-in Help}
Within the REPL you can use a built-in interactive documentation look-up feature.
This is roughly similar the interactive help features like in Python or R.

For this documentation look-up there exist two functions:
\begin{itemize}
  \item \bfix{doc}
  \item and \bfix{find-doc}
\end{itemize}

With ``\texttt{doc}'' you can lookup the documentation for a particular function, macro, special form, etc.
The syntax for using \texttt{doc} is as follows: \texttt{(doc <NAME>)}, with \texttt{<NAME>} being the entity you want to look up.
An example for looking up a documentation is shown in Listing \vref{lst:doc_lookup}.

\begin{lstlisting}[label=lst:doc_lookup, caption=Documentation Look-up]
user=> (doc +)
-------------------------
clojure.core/+
([] [x] [x y] [x y & more])
  Returns the sum of nums. (+) returns 0. Does not auto-promote
  longs, will throw on overflow. See also: +'
nil
user=>
\end{lstlisting}

In this example we looked up the documentation for the ``\texttt{+}'' function\footnote{Yes, in Clojure it is completely valid to name a function ``+''. Another interesting point might be that ``\texttt{+}'' is indeed a function and not something built into the language like is the case for C or Java.}.
For now, you will most probably not understand everything that is printed there.
Do not worry about this, as you learn more about Clojure you will step by step learn what all the things printed there mean.
Here, it is sufficient for you to memorize that Clojure offers nice tools if you are in need for help.
Note that when looking up macros and special forms, the output will mention that the particular entity is a macro or a special form.

With ``\texttt{find-doc}'' you can search in the documentation.
In the simplest case, you can simply supply a string that will be searched for.
An example of searching for a string is shown in Listing \vref{lst:find-doc_string}.
Please note that, for the sake of brevity, not the full output is shown there.
I put in ``\ldots'' to indicate that things had been left out.

\begin{lstlisting}[label=lst:find-doc_string, caption=Search the Documentation for a String]
user=> (find-doc "split")
-------------------------
clojure.core/partition-by
([f coll])
  Applies f to each value in coll, splitting it each time f returns
   a new value.  Returns a lazy seq of partitions.
-------------------------
...
-------------------------
clojure.string/split-lines
([s])
  Splits s on \n or \r\n.
nil
user=> 
\end{lstlisting}

A more sophisticate way to search the documentation is to use \bfix{regular expressions} (\bfix{regex}).
\texttt{find-doc} also accepts regex and will then search for these.
Assume you want to look up how to split strings you could, e.\,g., use something like this: ``\texttt{(find-doc \#"Split.*string")}''.
The result of this is shown in Listing \vref{lst:find-doc_regex}.

\begin{lstlisting}[label=lst:find-doc_regex, caption=Search the Documentation via Regex]
user=> (find-doc #"Split.*string")
-------------------------
clojure.string/split
([s re] [s re limit])
  Splits string on a regular expression.  Optional argument limit is
  the maximum number of splits. Not lazy. Returns vector of the splits.
nil
user=> 
\end{lstlisting}

Not that ``\#"something"'' is the Clojure shorthand to create a regular expression.
For now, you do not need to memorize this.
I will remember you how to create regex once it becomes important again.

\subsection{Clojure Cheat Sheet}
The Clojure Cheat Sheet is intended for quickly looking up the most important information about Clojure.
Essentially, it is a quick reference for the Clojure basics.
Personally, I found it to be a invaluable tool for quickly looking things up.

You can browse the Clojure Cheat Sheet online at the Clojure homepage \footnote{\url{http://clojure.org/cheatsheet}} or download it in PDF format from there: \url{https://github.com/jafingerhut/clojure-cheatsheets/blob/master/pdf/cheatsheet-usletter-color.pdf?raw=true}.
%FIXME: Add proper links!
I suggest to download the PDF version and store it somewhere on you computer, so you have it always with you.

\subsection{Online API Documentation}
On the Clojure homepage there is also a browsable API documentation.
This documentation is very useful if you are searching for certain functionality that goes beyond what is covered in the Clojure Cheat Sheet.

Additionally, you will find links to the source code of all the functions or macros that make up the Clojure API there.
This can be very handy if you like to peek into the code, e.\,g., to see how the real Clojure experts solve problems.

%TODO: Explain how to download the help from github?

\subsection{Books}
There also exists quite a number of books about Clojure.
Personally, I found ``Clojure in Action'' by Amit Rathore and ``The Joy of Clojure'' by Michael Fogus and Chris Houser, both available from Manning publications, very helpful for learning Clojure.

\section{Bye REPL!}
Last but not least you surely want to eventually close the REPL.
Recall that Clojure is running in a Java Virtual Machine (JVM).
So, for closing the REPL it is sufficient to shut the JVM down.

We can simply shut the JVM down via the static ``\texttt{exit}'' Method in the Java ``\texttt{System}'' class.
To do this simply type ``\texttt{(System/exit 0)}'' in the REPL as shown in Listing \vref{lst:repl_exit}.

\begin{lstlisting}[label=lst:repl_exit, caption=Quit the REPL]
user=> (System/exit 0)
[rc@colin clojure-by-example]$
\end{lstlisting}

You did this already before in section \vref{sec:test_clojure_installation} when you tested your Clojure installation,
but this time you know a little better what you are doing.
Don't worry about too many details for now, we will cover them bit by bit.

Closing the REPL is presumably the very first time where you make use of Clojures (very great) Java interoperability features.
In this example here you simply called a static method of a common Java class.
Clojure offers many great ways for interoperating with Java.

Essentially, you can use all Java code and libraries inside Clojure \footnote{I, e.\,g., use Clojures Java interoperability features in ``cljNetPcap'' (\url{https://github.com/ruedigergad/cljNetPcap}), a wrapper and convenience library for working with ``jNetPcap'' (\url{http://jnetpcap.com/}) from within Clojure.}.
In turn, you can also run Clojure code from within Java.
But, as you might have guessed, we will cover more details later.

